%%%%%%%%%%%%%%%%%%%%%%%%%%%%%%%%%%%%%%%%%%%%%%%%%%%%%%
%% Technical University of Munich
%% Institute for Electrical Energy Storage Technology
%% 
%% Version:    03. August 2020
%%%%%%%%%%%%%%%%%%%%%%%%%%%%%%%%%%%%%%%%%%%%%%%%%%%%%%

\chapter{Working with {\LaTeX}}
\label{cha:latex}
It is advisable to store the actual content or main part of the document in separate \texttt{.tex} files (subfolder \texdir). In order to be able to include these at the appropriate position, the commands \lstinline|\include{filename.tex}| (forces a new page) or \lstinline|\input{filename.tex}| (without page break) are used. Since the \gls{gls:ees} template uses the encoding \gls{gls:utf8}, make sure that this encoding is set when creating and saving new \texttt{.tex} files.

\input{TEX/Usage/LinePageBreak.tex}
\input{TEX/Usage/WhiteSpace.tex}
\input{TEX/Usage/Hyphenation.tex}
\input{TEX/Usage/Structure.tex}
\input{TEX/Usage/References.tex}
\input{TEX/Usage/Footnotes.tex}
\input{TEX/Usage/Citation.tex}
\input{TEX/Usage/Lists.tex}
\input{TEX/Usage/Objects.tex}
\input{TEX/Usage/Math.tex}
\input{TEX/Usage/Chemistry.tex}
\input{TEX/Usage/ListOf.tex}
\input{TEX/Usage/Miscellaneous.tex}