%%%%%%%%%%%%%%%%%%%%%%%%%%%%%%%%%%%%%%%%%%%%%%%%%%%%%%
%% Technical University of Munich
%% Institute for Electrical Energy Storage Technology
%% 
%% Version:    03. August 2020
%%%%%%%%%%%%%%%%%%%%%%%%%%%%%%%%%%%%%%%%%%%%%%%%%%%%%%

\section{List of standard packages (default packages)}
\label{app:packages}
\begin{description}
	\item[\href{https://www.ctan.org/pkg/amsfonts}{amssymb}] This file defines all the symbols found in the AMS symbol fonts msam and msbm. \cite{CTANTeam.2020bg}
	
	\item[\href{https://www.ctan.org/pkg/amsmath}{amsmath}] The principal package in the AMS-{\LaTeX} distribution. It adapts for use in {\LaTeX} most of the mathematical features found in AMS-{\TeX}; it is highly recommended as an adjunct to serious mathematical typesetting in {\LaTeX}. \cite{CTANTeam.2020r}
	
	\item[\href{https://www.ctan.org/pkg/amsthm}{amsthm}] The package facilitates the kind of theorem setup typically needed in American Mathematical Society publications. The package offers the theorem setup of the AMS document classes (amsart, amsbook, etc.) encapsulated in {\LaTeX} package form so that it can be used with other document classes. \cite{CTANTeam.2020ac}
	
	\item[\href{https://www.ctan.org/pkg/array}{array}] An extended implementation of the array and tabular environments which extends the options for column formats, and provides \enquote{programmable} format specifications. \cite{CTANTeam.2020bb}
	
	\item[\href{https://www.ctan.org/pkg/babel}{babel}] This package manages culturally-determined typographical (and other) rules for a wide range of languages. A document may select a single language to be supported, or it may select several, in which case the document may switch from one language to another in a variety of ways. \cite{CTANTeam.2020}
	
	\item[\href{https://www.ctan.org/pkg/biblatex}{biblatex}] Bib{\LaTeX} is a complete reimplementation of the bibliographic facilities provided by {\LaTeX}. Formatting of the bibliography is entirely controlled by {\LaTeX} macros, and a working knowledge of {\LaTeX} should be sufficient to design new bibliography and citation styles. Bib{\LaTeX} uses its own data backend program called \enquote{biber} to read and process the bibliographic data. With biber, Bib{\LaTeX} has many features rivalling or surpassing other bibliography systems. \cite{CTANTeam.2020ag}
	
	\item[\href{https://www.ctan.org/pkg/bm}{bm}] The bm package defines a command \verb|\bm| which makes its argument bold. The argument may be any maths object from a single symbol to an expression. This is closely related to the specification of the \verb|\bm| command in AMS-{\LaTeX}, but \verb|\bm| is rather more careful in the way it does things. \cite{CTANTeam.2020bh}
	
	\item[\href{https://www.ctan.org/pkg/booktabs}{booktabs}] The package enhances the quality of tables in {\LaTeX}, providing extra commands as well as behind-the-scenes optimisation. Guidelines are given as to what constitutes a good table in this context. From version 1.61, the package offers longtable compatibility. \cite{CTANTeam.2020bc}
	
	\item[\href{https://www.ctan.org/pkg/caption}{caption}] The caption package provides many ways to customise the captions in floating environments like figure and table, and cooperates with many other packages. Facilities include rotating captions, sideways captions, continued captions (for tables or figures that come in several parts). A list of compatibility notes, for other packages, is provided in the documentation. The package also provides the \enquote{caption outside float} facility, in the same way that simpler packages like capt-of do. \cite{CTANTeam.2020ar}
	
	\item[\href{https://www.ctan.org/pkg/chngcntr}{chngcntr}] Defines commands \verb|\counterwithin| (which sets up a counter to be reset when another is incremented) and \verb|\counterwithout| (which unsets such a relationship). \cite{CTANTeam.2020b}
	
	\item[\href{https://www.ctan.org/pkg/circuitikz}{circuitikz}] The package provides a set of macros for naturally typesetting electrical and (somewhat less naturally, perhaps) electronic networks. \cite{CTANTeam.2020ax}
	
	\item[\href{https://www.ctan.org/pkg/color}{color}] The color package provides both foreground (text, rules, etc.) and background colour management; it uses the device driver configuration mechanisms of the graphics package to determine how to control its ouptut. \cite{CTANTeam.2020at}
	
	\item[\href{https://www.ctan.org/pkg/csquotes}{csquotes}] This package provides advanced facilities for inline and display quotations. It is designed for a wide range of tasks ranging from the most simple applications to the more complex demands of formal quotations. The facilities include commands, environments, and user-definable \enquote{smart quotes} which dynamically adjust to their context. Quotation marks are switched automatically if quotations are nested and they can be adjusted to the current language if the babel package is available. There are additional facilities designed to cope with the more specific demands of academic writing, especially in the humanities and the social sciences. All quote styles as well as the optional active quotes are freely configurable. \cite{CTANTeam.2020am}
	
	\item[\href{https://www.ctan.org/pkg/enumitem}{enumitem}] This package provides user control over the layout of the three basic list environments: enumerate, itemize and description. It supersedes both enumerate and mdwlist (providing well-structured replacements for all their funtionality), and in addition provides functions to compute the layout of labels, and to \enquote{clone} the standard environments, to create new environments with counters of their own. \cite{CTANTeam.2020g}
	
	\item[\href{https://www.ctan.org/pkg/epigraph}{epigraph}] Epigraphs are the pithy quotations often found at the start (or end) of a chapter. Both single epigraphs and lists of epigraphs are catered for. Various aspects are easily configurable. \cite{CTANTeam.2020ap}
	
	\item[\href{https://www.ctan.org/pkg/etoolbox}{etoolbox}] The package is a toolbox of programming facilities geared primarily towards {\LaTeX} class and package authors. It provides {\LaTeX} frontends to some of the new primitives provided by e-{\TeX} as well as some generic tools which are not strictly related to e-{\TeX} but match the profile of this package. Note that the initial versions of this package were released under the name elatex. \cite{CTANTeam.2020as}
	
	\item[\href{https://www.ctan.org/pkg/fancyvrb}{fancyvrb}] Flexible handling of verbatim text including: verbatim commands in footnotes; a variety of verbatim environments with many parameters; ability to define new customized verbatim environments; save and restore verbatim text and environments; write and read files in verbatim mode; build \enquote{example} environments (showing both result and verbatim source). \cite{CTANTeam.2020aq}
	
	\item[\href{https://www.ctan.org/pkg/float}{float}] Improves the interface for defining floating objects such as figures and tables. Introduces the boxed float, the ruled float and the plaintop float. You can define your own floats and improve the behaviour of the old ones. The package also provides the H float modifier option of the obsolete here package. You can select this as automatic default with \verb|\floatplacement{figure}{H}|. \cite{CTANTeam.2020k}
	
	\item[\href{https://www.ctan.org/pkg/fontenc}{fontenc}] The package allows the user to select font encodings, and for each encoding provides an interface to \enquote{font-encoding-specific} commands for each font. Its most powerful effect is to enable hyphenation to operate on texts containing any character in the font. \cite{CTANTeam.2020aj}
	
	\item[\href{https://www.ctan.org/pkg/fontspec}{fontspec}] Fontspec is a package for {\XeLaTeX} and {\LuaLaTeX}. It provides an automatic and unified interface to feature-rich AAT and OpenType fonts through the NFSS in {\LaTeX} running on {\XeTeX} or {\LuaTeX} engines. \cite{CTANTeam.2020bo}
	
	\item[\href{https://www.ctan.org/pkg/footnote}{footnote}] The footnote package provides commands for handling footnotes rather more fluently than {\LaTeX} manages. Several {\LaTeX} commands and environments (for example, \verb|\parbox|, minipage and tabular) \enquote{trap} footnotes. The footnote package provides an environment savenotes, that can be wrapped around a command or environment; at \verb|\end{savenotes}| all the footnotes within will emerge. The command \verb|\makesavenoteenv| will generate environments that behave as if they've been wrapped in a savenotes environment. \cite{CTANTeam.2020c}
		
	\item[\href{https://www.ctan.org/pkg/geometry}{geometry}] The package provides an easy and flexible user interface to customize page layout, implementing auto-centering and auto-balancing mechanisms so that the users have only to give the least description for the page layout. For example, if you want to set each margin 2cm without header space, what you need is just \verb|\usepackage[margin=2cm,nohead]{geometry}|. \cite{CTANTeam.2020al}
	
	\item[\href{https://www.ctan.org/pkg/glossaries-extra}{glossaries-extra}] This package provides improvements and extra features to the glossaries package. Some of the default glossaries.sty behaviour is changed by glossaries-extra.sty. See the user manual glossaries-extra-manual.pdf for further details. \cite{CTANTeam.2020ah}
	
	\item[\href{https://www.ctan.org/pkg/graphicx}{graphicx}] The package builds upon the graphics package, providing a key-value interface for optional arguments to the \verb|\includegraphics| command. This interface provides facilities that go far beyond what the graphics package offers on its own. \cite{CTANTeam.2020j}
	
	\item[\href{https://www.ctan.org/pkg/hyperref}{hyperref}] The hyperref package is used to handle cross-referencing commands in {\LaTeX} to produce hypertext links in the document. \cite{CTANTeam.2020u}
	
	\item[\href{https://www.ctan.org/pkg/ifthen}{ifthen}] Conditional commands in {\LaTeX} documents. The package's basic command is \verb|\ifthenelse|, which can use a wide array of tests. Also provided is a simple loop command \verb|\whiledo|. \cite{CTANTeam.2020ai}
	
	\item[\href{https://www.ctan.org/pkg/import}{import}] The commands \verb|\import{full_path}{file}| and \verb|\subimport{path_extension}{file}| set up input through standard {\LaTeX} mechanisms (\verb|\input|, \verb|\include| and \verb|\includegraphics|) to load files relative to the \verb|\import|-ed directory. \cite{CTANTeam.2020av}
	
	\item[\href{https://www.ctan.org/pkg/letltxmacro}{letltxmacro}] {\TeX}'s \verb|\let| assignment does not work for {\LaTeX} macros with optional arguments or for macros that are defined as robust macros by \verb|\DeclareRobustCommand|. This package defines \verb|\LetLtxMacro| that also takes care of the involved internal macros. \cite{CTANTeam.2020bf}
	
	\item[\href{https://www.ctan.org/pkg/lettrine}{lettrine}] \textsc{Only for ees-article.} The lettrine package supports various dropped capitals styles, typically those described in the French typographic books. In particular, it has facilities for the paragraph text's left edge to follow the outline of capitals that have a regular shape (such as \enquote{A} and \enquote{V}). \cite{CTANTeam.2020bw}
	
	\item[\href{https://www.ctan.org/pkg/lineno}{lineno}] Adds line numbers to selected paragraphs with reference possible through the {\LaTeX} \verb|\ref| and \verb|\pageref| cross reference mechanism. Line numbering may be extended to footnote lines, using the fnlineno package. \cite{CTANTeam.2020bv}
	
	\item[\href{https://www.ctan.org/tex-archive/info/lmodern}{lmodern}] The Latin Modern fonts, also known as \enquote{lm fonts}, are a set of scalable fonts in PostScript Type 1 and OpenType formats. They are based on the PostScript Type 1 version of the Computer Modern fonts and contain many additional characters (mostly accented ones). This package provides {\TeX} support and Type1 (PostScript) fonts. If only the OpenType fonts are needed, please see the package fonts-lmodern. \cite{CTANTeam.2020bn}
	
	\item[\href{https://www.ctan.org/pkg/lstaddons}{lstautogobble}] This add-on package to listings provides a boolean autogobble setting which will automatically set the gobble setting to indention of the first line. \cite{CTANTeam.2020bd}
	
	\item[\href{https://www.ctan.org/pkg/listings}{listings}] The package enables the user to typeset programs (programming code) within {\LaTeX}; the source code is read directly by {\TeX} -- no front-end processor is needed. Keywords, comments and strings can be typeset using different styles (default is bold for keywords, italic for comments and no special style for strings). Support for hyperref is provided. \cite{CTANTeam.2020z}
	
	\item[\href{https://www.ctan.org/pkg/luainputenc}{luainputenc}] Lua{\TeX} operates by default in UTF-8 input; thus {\LaTeX} documents that need 8-bit character-sets need special treatment. (In fact, {\LaTeX} documents using UTF-8 with \enquote{traditional} -- 256-glyph -- fonts also need support from this package.). \cite{CTANTeam.2020ak}
	
	\item[\href{https://www.ctan.org/pkg/mathtools}{mathtools}] Mathtools provides a series of packages designed to enhance the appearance of documents containing a lot of mathematics. The main backbone is amsmath, so those unfamiliar with this required part of the {\LaTeX} system will probably not find the packages very useful. \cite{CTANTeam.2020bk}
	
	\item[\href{https://www.ctan.org/pkg/matlab-prettifier}{matlab-prettifier}] The package extends the facilities of the listings package, to pretty-print Matlab and Octave source code. (Note that support of Octave syntax is not complete.) \cite{CTANTeam.2020ab}
	
	\item[\href{https://www.ctan.org/pkg/mhchem}{mhchem}] The bundle provides three packages: 1) The mhchem package provides commands for typesetting chemical molecular formulae and equations. 2) The hpstatement package provides commands for the official hazard statements and precautionary statements (H and P statements) that are used to label chemicals. 3) The rsphrase package provides commands for the official Risk and Safety (R and S) Phrases that are used to label chemicals. \cite{CTANTeam.2020t}
	
	\item[\href{https://www.ctan.org/pkg/multirow}{multirow}] The package has a lot of flexibility, including an option for specifying an entry at the \enquote{natural} width of its text. The package is distributed with the bigdelim and bigstrut packages, which can be used to advantage with \verb|\multirow| cells. \cite{CTANTeam.2020q}
	
	\item[\href{https://www.ctan.org/pkg/pdfpages}{pdfpages}] This package simplifies the inclusion of external multi-page PDF documents in {\LaTeX} documents. Pages may be freely selected and similar to psnup it is possible to put several logical pages onto each sheet of paper. Furthermore a lot of hypertext features like hyperlinks and article threads are provided. The package supports pdf{\TeX} (pdf{\LaTeX}) and V{\TeX}. With V{\TeX} it is even possible to use this package to insert PostScript files, in addition to PDF files. \cite{CTANTeam.2020ay}
	
	\item[\href{https://www.ctan.org/pkg/pgf}{pgf}] PGF is a macro package for creating graphics. It is platform- and format-independent and works together with the most important {\TeX} backend drivers, including pdf{\TeX} and dvips. It comes with a user-friendly syntax layer called TikZ. \cite{CTANTeam.2020m}
	
	\item[\href{https://www.ctan.org/pkg/pgfplots}{pgfplots}] PGF is a macro package for creating graphics. It is platform- and format-independent and works together with the most important {\TeX} backend drivers, including pdf{\TeX} and dvips. It comes with a user-friendly syntax layer called TikZ. \cite{CTANTeam.2020m}
	
	\item[\href{https://www.ctan.org/pkg/pgfplotstable}{pgfplotstable}] Pgfplotstable displays numerical tables rounded to desired precision in various display formats (for example scientific format, fixed point format or integer), using {\TeX}'s mathematical facilities for pretty printing. Furthermore, it provides methods for table postprocessing. \cite{CTANTeam.2020aw}
	
	\item[\href{https://www.ctan.org/pkg/prelim2e}{prelim2e}] Puts text below the normal page content (the default text marks the document as draft and puts a timestamp on it). Can be used together with e.g. the vrsion, rcs and rcsinfo packages. Uses the everyshi package and can use the scrtime package from the koma-script bundle. \cite{CTANTeam.2020bl}
	
	\item[\href{https://www.ctan.org/pkg/realboxes}{realboxes}] The package uses the author's package collectbox to define variants of common box related macros which read the content as real box and not as macro argument. This enables the use of verbatim or other special material as part of this content. \cite{CTANTeam.2020be}
	
	\item[\href{https://www.ctan.org/pkg/rotating}{rotating}] A package built on the standard {\LaTeX} graphics package to perform all the different sorts of rotation one might like, including complete figures and tables with their captions. If you want continuous text (i.e., more than one page) set in landscape mode, use the lscape package instead. The rotating packages only deals in rotated boxes (or floats, which are themselves boxes), and boxes always stay on one page. \cite{CTANTeam.2020az}
	
	\item[\href{https://www.ctan.org/pkg/koma-script}{scrhack}] scrhack is a {\LaTeX} package of the KOMA-Script bundle. It patches other packages to make them work better, add new features to them and to improve their interaction with KOMA-Script. One main feature is, to make them work with tocbasic instead of the KOMA-Script's deprecated float list interface. Currently patches for float.sty, floatrow.sty, (old versions of) hyperref, listings, and setspace are available. \cite{CTANTeam.2020bs}
	
	\item[\href{https://www.ctan.org/pkg/scrlayer-scrpage}{scrlayer-scrpage}] This package is the part of the Koma-Script bundle that provides an end user interface to scrlayer, allowing the user to define and manage page styles by controlling page headers and footers. \cite{CTANTeam.2020bt}
	
	\item[\href{https://www.ctan.org/pkg/setspace}{setspace}] Provides support for setting the spacing between lines in a document. Package options include singlespacing, onehalfspacing, and doublespacing. Alternatively the spacing can be changed as required with the \verb|\singlespacing|, \verb|\onehalfspacing|, and \verb|\doublespacing| commands. Other size spacings also available. \cite{CTANTeam.2020bu}
	
	\item[\href{https://www.ctan.org/pkg/showframe}{showframe}] The package shows a (simple, uncluttered) diagram of the page layout; similar (but more complex-looking) diagrams may be obtained from the layouts and geometry packages. \cite{CTANTeam.2020bm}
	
	\item[\href{https://www.ctan.org/pkg/siunitx}{siunitx}] Typesetting values with units requires care to ensure that the combined mathematical meaning of the value plus unit combination is clear. In particular, the SI units system lays down a consistent set of units with rules on how they are to be used. \cite{CTANTeam.2020s}
	
	\item[\href{https://www.ctan.org/pkg/subcaption}{subcaption}] The package provides a means of using facilities analagous to those of the caption package, when writing captions for subfigures and the like. The package is distributed with caption. \cite{CTANTeam.2020l}
	
	\item[\href{https://www.ctan.org/pkg/tabularx}{tabularx}] The package defines an environment tabularx, an extension of tabular which has an additional column designator, X, which creates a paragraph-like column whose width automatically expands so that the declared width of the environment is filled. (Two X columns together share out the available space between them, and so on.) \cite{CTANTeam.2020p}
	
	\item[\href{https://www.ctan.org/pkg/tcolorbox}{tcolorbox}] This package provides an environment for coloured and framed text boxes with a heading line. Optionally, such a box may be split in an upper and a lower part; thus the package may be used for the setting of {\LaTeX} examples where one part of the box displays the source code and the other part shows the output. Another common use case is the setting of theorems. The package supports saving and reuse of source code and text parts. \cite{CTANTeam.2020bq}
	
	\item[\href{https://www.ctan.org/pkg/tikz}{tikz}] See package pgf.
	
	\item[\href{https://www.ctan.org/pkg/todonotes}{todonotes}] The package lets the user mark things to do later, in a simple and visually appealing way. The package takes several options to enable customization/finetuning of the visual appearance. \cite{CTANTeam.2020x}
	
	\item[\href{https://www.ctan.org/pkg/threeparttable}{threeparttable}] Provides a scheme for tables that have a structured note section, after the caption. This scheme provides an answer to the old problem of putting footnotes in tables -- by making footnotes entirely unnecessary. Note that a threeparttable is not a float of itself; but you can place it in a table or a table* environment, if necessary. \cite{CTANTeam.2020f}
	
	\item[\href{https://www.ctan.org/tex-archive/macros/latex/contrib/caption/}{totalcount}] This package offers commands for typesetting total values of counters. \cite{CTANTeam.2020bp}
	
	\item[\href{https://www.ctan.org/pkg/upgreek}{upgreek}] A package to provide the upright Greek letters from the Euler or Adobe Symbol fonts as additional math symbols (\verb|\upalpha|, \verb|\upbeta|, etc.), with proper spacing and scaling in super- and subscripts. \cite{CTANTeam.2020bi}
	
	\item[\href{https://www.ctan.org/pkg/xcolor}{xcolor}] The package starts from the basic facilities of the color package, and provides easy driver-independent access to several kinds of color tints, shades, tones, and mixes of arbitrary colors. It allows a user to select a document-wide target color model and offers complete tools for conversion between eight color models. Additionally, there is a command for alternating row colors plus repeated non-aligned material (like horizontal lines) in tables. Colors can be mixed like \verb|\color{red!30!green!40!blue}|. \cite{CTANTeam.2020au}
	
	\item[\href{https://www.ctan.org/pkg/xfrac}{xfrac}] The package allows the user to typeset fractions in the form \enquote{n/d}. Requires e-{\TeX}, current {\LaTeX}, and the {\LaTeX}3 packages template and xparse. The package provides a high quality alternative to the facilities of the nicefrac package (part of the units distribution). See also the faktor package for quotient spaces and the like. \cite{CTANTeam.2020bj}
	
	\item[\href{https://www.ctan.org/pkg/xparse}{xparse}] The package provides a high-level interface for producing document-level commands. In that way, it offers a replacement for \LaTeXe's \verb|\newcommand| macro, with significantly improved functionality. \cite{CTANTeam.2020br}
	
	\item[\href{https://www.ctan.org/pkg/xspace}{xspace}] The xspace package provides a single command that looks at what comes after it in the command stream, and decides whether to insert a space to replace one \enquote{eaten} by the {\TeX} command decoder. The decision is based on what came after any space, not on whether there was a space (which is unknowable): so if the next thing proves to be punctuation, the chances are there was no space, but if it's a letter, there's probably a need for space. This technique is not perfect, but works in a large proportion of cases. \cite{CTANTeam.2020an}
	
	\item[\href{https://www.ctan.org/pkg/xurl}{xurl}] This package loads [the package] url by default and defines possible URL breaks for all alphanumerical characters, as well as = / . : * - \textasciitilde{} ' ". \cite{CTANTeam.2020ao}
\end{description}