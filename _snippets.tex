%%%----- Custum Commands ------------------------------
\newcommand{\eg}{\mbox{e.g.}\xspace}
\newcommand{\ie}{\mbox{i.e.}\xspace}
\newcommand{\etc}{\mbox{etc.}\xspace}

\newcommand{\matlab}{MATLAB\textregistered\xspace}
\newcommand{\tum}{Technische Universität München\xspace}

\newcommand{\mainfilename}{EES\_xyz}
\newcommand{\mainfile}{\texttt{\mainfilename.tex}\xspace}
\newcommand{\classfile}{\texttt{ees.cls}\xspace}
\newcommand{\acrfile}{\texttt{./LIST/acronyms.tex}\xspace}
\newcommand{\glofile}{\texttt{./LIST/glossaries.tex}\xspace}
\newcommand{\symfile}{\texttt{./LIST/symbols.tex}\xspace}
\newcommand{\litfile}{\texttt{./LIST/literature.bib}\xspace}
\newcommand{\hypfile}{\texttt{./LIST/hyphenation.tex}\xspace}

\newcommand{\logodir}{\texttt{./logo/}\xspace}
\newcommand{\imgdir}{\texttt{./IMG/}\xspace}
\newcommand{\listdir}{\texttt{./LIST/}\xspace}
\newcommand{\texdir}{\texttt{./TEX/}\xspace}
\newcommand{\tikzdir}{\texttt{./TIKZ/}\xspace}

\newcommand{\Latex}{\LaTeX\xspace}

\providecommand{\ph}[1]{\relax}
%\renewcommand{\ph}[1]{\textlangle#1\textrangle\xspace}				% placeholder.
\renewcommand{\ph}[1]{{\normalfont\guilsinglleft}#1{\normalfont\guilsinglright}\xspace}	% placeholder.


\usepackage[utf8]{luainputenc}

\usepackage{blindtext}

\usepackage{textalpha}

\usepackage{metalogo} % For use of \XeLaTeX


\lstset{
	style=defaultCode,
	language=latex,
}

\lstdefinestyle{mweCode}{
	style=defaultCode,
	basicstyle=\ttfamily\scriptsize,
	numbers=none,
}

\newtheorem{definition}{Definition}
\newcommand{\definitionautorefname}{Definition}

% https://tex.stackexchange.com/a/39020
\renewcommand{\bottomfraction}{0.5} % (default 0.3) maximum size of the bottom area for float environments

\DeclareSIUnit{\Crate}{C\textsubscript{rate}}



% Allows 'drop caps' like in IEEE papers
\RequirePackage{lettrine}
\renewcommand{\LettrineFontHook}{\fontseries{bx}}	% Change font to bold series (italic with \fontshape{sl})
\setcounter{DefaultLines}{2}	% First letter will occupy two lines
\renewcommand*{\DefaultLhang}{0.0}	% First letter will not hang into the margin (values between 0 and 1)
\setlength{\DefaultFindent}{0.25em}	% (positive or negative) controls the horizontal gap between the dropped capital and the indented block of text (default=0pt)
\setlength{\DefaultNindent}{0.0em}	% shifts all indented lines, starting from the second one, horizontally by <dimen> (this shift is relative to the first line, default=0.5em)


%%% ----- Folder and File Picture --------------------
% Taken from: https://tex.stackexchange.com/a/405253
\definecolor{folderbg}{RGB}{124,166,198}
\definecolor{folderborder}{RGB}{110,144,169}
\newlength\Size
\setlength\Size{4pt}
\tikzset{%
	folder/.pic={%
		\filldraw [draw=folderborder, top color=folderbg!50, bottom color=folderbg] (-1.05*\Size,0.2\Size+5pt) rectangle ++(.75*\Size,-0.2\Size-5pt);
		\filldraw [draw=folderborder, top color=folderbg!50, bottom color=folderbg] (-1.15*\Size,-\Size) rectangle (1.15*\Size,\Size);
	},
	file/.pic={%
		\filldraw [draw=folderborder, top color=folderbg!5, bottom color=folderbg!10] (-\Size,.4*\Size+5pt) coordinate (a) |- (\Size,-1.2*\Size) coordinate (b) -- ++(0,1.6*\Size) coordinate (c) -- ++(-5pt,5pt) coordinate (d) -- cycle (d) |- (c) ;
	},
}


%%% ----- Renew maketitle command for special title
\renewcommand{\maketitle}{%
	\begin{titlepage}%
		\thispagestyle{empty}%
		\makeheader%
		\vspace{10mm}%
		\begin{center}%
			\sffamily
			\begin{spacing}{2}%
				\huge{\textbf{\printtitle}}\par%
				\Large{\textbf{\printsubtitle}}%
			\end{spacing}%
			\vspace{20mm}%
			\begin{spacing}{1}%
				\Large{\printtopic}\par%
				\printfrom \par%
				\Large{\printauthor}%
			\end{spacing}%
			\vspace{10mm}
			\begin{spacing}{1}%
				\Large{Based on Work}\par%
				\printfrom \par%
				\Large{Max Horsche}%
			\end{spacing}%
			\vfill%
			\ifthenelse{\equal{\@supervisor}{true}}{
				\begin{spacing}{2}%
					\Large{\printadvisor}\par%
					\Large{\printsupervisor}%
				\end{spacing}%
			}{}
			\vspace{20mm}%
			{\Large{\printdate}}%
		\end{center}%
	\end{titlepage}%
	\pagenumbering{Roman}%
	\setcounter{page}{1}%
}


%%%%----- PGF plot settings ---------------------------
%\pgfplotsset{plot coordinates/math parser=false}
%\newlength\figureheight
%\newlength\figurewidth
%\pgfplotsset{width=10cm,height=5cm}
%
%\pgfplotsset{
%	scale only axis,									% Nur Achsen skalieren, keine Beschriftungen usw.
%	width=7cm,												% Breite der Grafik
%	%height=3cm,											% Höhe der Grafik
%	%% Globale Achsen Eigenschaften
%	every axis/.append style={
%		line width=1pt,									% Dicke von Achsen und Plots
%		tick style={line width=0.8pt},	% Dicke der Ticksstriche
%		grid style={dashed, Schwarz!20},	% Grid gestrichelt in grau
%		grid=major,											% Haupt-Grid anzeigen
%	},
%	%% Globale Legenden Eigenschaften
%	every axis legend/.append style={
%		at={(0.5,1.05)}, anchor=south,	% Mittig über Graphen platzieren
%		draw=Schwarz,										% Rahmenfarbe
%		line width=0.4pt,								% Liniendicke betrifft ALLE!
%		cells={line width=1pt},					% Liniendicke betrifft NUR Legendeineinträge
%		fill=Weiss,											% Füllfarbe
%		legend cell align=left,					% Linksbündig
%	},
%}
%
%%%%----- Tikz Erweiterungen --------------------------
%\usetikzlibrary{shapes, arrows} % arrows, automata, backgrounds, calendar, chains, matrix, mindmap, patterns, petri, shadows, shapes.geometric, shapes.misc, spy, trees
%	
%
%%%%----- Globale Tikz Einstellungen ------------------
%\tikzset{>=latex}
%\tikzset{
%%% Globale Pin Eigenschaften
%	every pin/.append style={
%		fill=Elfenbein,									% Füllfarbe
%		rectangle,											% Form der Markierung
%		%rounded corners=1pt,						% Abgerundete Ecken
%		draw=Schwarz,										% Rahmenfarbe
%		line width=0.4pt,								% Liniendicke betrifft ALLE!
%		font=\footnotesize,							% Schriftgröße
%		pin distance=0.3cm							% Pin-Abstand zur Markierung
%	},
%	%% Globale Markierungs Eigenschaften
%	small dot/.append style={
%		fill=Schwarz,										% Füllfarbe
%		circle,													% Punkt
%		scale=0.5,											% Größe
%	},
%}
%
%%%%----- Show current fontsize -----------------------
%\makeatletter
%\newcommand\thefontsize[1]{{#1 Current font size is: \f@size pt\par}}
%\makeatother
%
%%%%----- Karokästchen --------------------------------
%\newcommand\kariert[2][0.5cm]{% 
%   \begin{tikzpicture}[gray,step=#1]
%     \pgfmathtruncatemacro\anzahl{(\linewidth-\pgflinewidth)/#1} % maximale Anzahl Kästchen pro Zeile
%     \draw (0,0) rectangle (\anzahl*#1,#2*#1) (0,0) grid (\anzahl*#1,#2*#1);
%   \end{tikzpicture} 
%}
%
%%%%----- Spaltenbreiten ------------------------------
%\newcommand*\showwidth[1]{%
%	Current \texttt{\textbackslash#1} is: \expandafter\the\csname#1\endcsname\par
%	\vspace{-1em}%
%  \textcolor{blue}{\rule{\csname#1\endcsname}{1pt}}\par
%}