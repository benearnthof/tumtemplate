%%%%%%%%%%%%%%%%%%%%%%%%%%%%%%%%%%%%%%%%%%%%%%%%%%%%%%
%% Technical University of Munich
%% Institute for Electrical Energy Storage Technology
%% 
%% Version:    03. August 2020
%%%%%%%%%%%%%%%%%%%%%%%%%%%%%%%%%%%%%%%%%%%%%%%%%%%%%%

\section{Minimal examples of the available classes}
\label{app:samples}
Minimal examples are given here for the available class options \texttt{class}. For more information on each class option, see \autoref{sec:documentTypes}. 

A smaller font is chosen for better clarity and to fit the examples on one page. In addition, the line numbers have been removed, which makes it possible to copy and paste them into your own editor.


\clearpage
\subsection{Minimal Example -- ees-book}
\label{sec:app:samples:book}
\lstinputlisting[
	caption={Minimal Example -- \textit{ees-book} \label{lst:app:samples:book}}, 
	style=mweCode,
	]{./TEX/Samples/EES_Book.tex}


\clearpage
\subsection{Minimal Example -- ees-report}
\label{sec:app:samples:report}
\lstinputlisting[
	caption={Minimal Example -- \textit{ees-report} \label{lst:app:samples:report}}, 
	style=mweCode,
	]{./TEX/Samples/EES_Report.tex}



\clearpage
\subsection{Minimal Example -- ees-protocol}
\label{sec:app:samples:protocol}
\lstinputlisting[
	caption={Minimal Example -- \textit{ees-protocol} \label{lst:app:samples:protocol}}, 
	style=mweCode,
	]{./TEX/Samples/EES_Protocol.tex}



\clearpage
\subsection{Minimal Example -- ees-article}
\label{sec:app:samples:article}
\lstinputlisting[
	caption={Minimal Example -- \textit{ees-article} \label{lst:app:samples:article}}, 
	style=mweCode,
	]{./TEX/Samples/EES_Article.tex}